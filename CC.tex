High-performance computing (HPC) environments were designed to primarily
support the execution of single simulations. Current HPC platforms enable the
strong and weak scaling of single tasks (hitherto mostly simulations), with
limited software and systems support for the concurrent execution of multiple
heterogeneous tasks as part of a single application (or workflow). As the
nature of scientific inquiry and the applications to support that inquiry
evolve, there is a critical need to support the scalable and concurrent
execution of a large number of heterogeneous tasks.

Sets of tasks with dependencies that determine the order of their execution
are usually referred to as ``workflows''. Often times, the structure of the
task dependencies is simple and adheres to discernible patterns, even though
the individual tasks and their duration are non-trivially distinct. Put
together, it is a challenge to support the scalable execution of workflows on
HPC resources due to the existing software ecosystem and runtime systems
typically found.

Many workflow systems have emerged in response to the aforementioned problem.
Each workflow system has its strengths and unique capability, however each
system typically introduces its problems and challenges. In spite of the many
successes of workflow systems, there is a perceived high barrier-to-entry,
integration overhead and limited flexibility.

Interestingly, many commonly used workflow systems in high-performance and
distributed computing emerged from an era when the software landscape
supporting distributed computing was fragile, missing features and services.
Not surprisingly, initial workflow systems had a monolithic design that
included the end-to-end capabilities needed to execute workflows on
heterogeneous and distributed cyberinfrastructures. Further, these workflow
systems were typically designed by a set of specialists to support large
``big science'' projects such as those carried out at the
LHC~\cite{breskin2009cern} or LIGO~\cite{althouse1992ligo}. The fact that the
same workflow would be used by thousands of scientists over many years
justified, if not amortized, the large overhead of integrating application
workflows with monolithic workflow systems. This influenced the design and
implementation of interfaces and programming models.

However as the nature, number and usage of workflows has evolved so have the
requirements: scale remains important but only when delivered with the
ability to prototype quickly and flexibly. Furthermore, there are also new
performance requirements that arise from the need to support concurrent
execution of heterogeneous tasks. For example, when executing multiple
homogeneous pipelines of heterogeneous tasks, for reasons of efficient
resource utilization there is a need to ensure that the individual pipelines
have similar execution times. The pipeline-to-pipeline fluctuation must be
minimal while also managing the task-to-task runtime fluctuation across
concurrently executing pipelines.

% \mtnote{Should we define what an ensemble is?} \jdnote{We defined ensemble
% in the Methodology Section: We term this approach ensemble molecular
% dynamics, “ensemble” here referring to the set of individual (replica)
% simulations conducted for the same physical system.}

Thus the flexible execution of heterogeneous ensembles MD simulations face
both system software and middleware challenges: existing system software that
is typically designed to support the execution of single large simulations on
the one hand, and workflow systems that are designed to support specific use
cases or `locked-in' end-to-end executions. In the next Section, we discuss
the design and implementation of the RADICAL-Cybertools, a set of software
building blocks that can be easily composed to design, implement and execute
domain specific workflows rapidly and at scale.