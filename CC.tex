
% \subsection{Computational Challenges}

Supercomputers and high-performance computing environments were designed to
primarily support the execution of single simulations. More precisely, current
supercomputers enable the strong and weak scaling of single tasks (hitherto
mostly simulations), with limited software and systems support for the
concurrent execution of multiple heterogeneous tasks as part of a single
application (or workflow). However as the nature of scientific inquiry that
HPC are used for, as well as the programs and applications to support that
inquiry has evolved, there is a critical need to support the scalable and
concurrent execution of a large number of tasks, each of which is typically
smaller than the traditional single simulation.

The aggregation of these tasks and their dependencies are often referred to as
``workflows''. Often times the structure of the task dependencies is simple and
adheres to discernible patterns, even though the individual tasks and their
duration are non-trivially distinct. Put together, it is a challenge to
support the scalable execution of workflows on HPC resources due to the
existing software ecosystem and runtime systems typically found.

Many workflow systems have emerged over the years in response to the
aforementioned problem. Each workflow system has its strengths and unique
capability, however each system typically introduces its own set of problems
and challenges. In spite of the many successes of workflow systems, there is a
perceived high barrier-to-entry, integration overhead and limited flexibility.

Interestingly, many commonly used workflow systems in high-performance and
distributed computing emerged from an era when the software landscape
supporting distributed computing was fragile, missing features and services.
Not surprisingly, initial workflow systems, had a monolithic design that
included the end-to-end capabilities needed to execute workflows on
heterogeneous and distributed cyberinfrastructures. Further, these workflow
systems were typically designed by a set of specialists to support large ``big
science'' projects such as those carried out at the LHC or LIGO. The fact that
the same workflow would be used by thousands of scientists over many years
justified, if not amortized the large overhead of integrating application
workflows with monolithic workflow systems.

However as the nature, number and usage of workflows has evolved so have the
requirements: scale remains important but only when delivered with the ability
to prototype quickly and flexibly. Furthermore, there are also new performance
requirements that arise from the need to support concurrent execution of
heterogeneous tasks. For example, when executing multiple homogeneous
pipelines of heterogeneous tasks, for reasons of efficient resource
utilization there is a need to ensure that the individual pipelines have
similar execution times. The pipeline-to-pipeline fluctuation must be minimal
while also managing the task-to-task runtime fluctuation across concurrently
executing pipelines.

\begin{itemize}
	\item $N_T$ : Number of tasks
	\item ${{n_C}^T}$: Number of cores per task
	\item $N_c$	: Number of cores (total)
	\item $t_x$ : Time durations of a task
\end{itemize}






