The strength of drug binding is determined by a thermodynamic property known
as the binding free energy (or binding affinity). One promising technology for
estimating binding free energies and the influence of protein composition on
them is molecular dynamics (MD)~\cite{Karplus2005}. Our previous work
\cite{Sadiq2010, Wan2011} has demonstrated that running multiple MD
simulations based on the same system and varying only in initial velocities
offers a highly efficient method of obtaining accurate and reproducible
estimates of the binding affinity. We term this approach ensemble molecular
dynamics, ``ensemble'' here referring to the set of individual (replica)
simulations conducted for the same physical system. In this Section we discuss
the advantages to this approach.

% ---------------------------------------------------------------------------
\subsection{Ensemble Molecular Dynamics Simulations}

Atomistically detailed models of the drug and target protein can be used as the
starting point for MD simulations to
study the influence of mutations on drug binding. The chemistry of the system
is encoded in what is known as a potential~\cite{Karplus2002}. In the
parameterization of the models, each atom is assigned a mass and a charge,
with the chemical bonds between them modeled as springs with varying
stiffness. Using Newtonian mechanics the dynamics of the protein and drug can
be followed and, using the principles of statistical mechanics, estimates of
thermodynamic properties obtained from simulations of single particles.

The potentials used in the simulations are completely under the control of
the user. This allows the user to manipulate the system in ways which would
not be possible in experiments. A particularly powerful example of this are
the so called ``alchemical'' simulations in which the potential used in the
simulation changes, from representing a particular starting system into one 
describing a related target system during execution. This allows for the 
calculation of free energy differences between the two systems, such as those 
induced by a protein mutation.

MD simulations can reveal how interactions change as a result of mutations,
and account for the molecular basis of drug efficacy. This understanding can
form the basis for structure-based drug design as well as helping to target
existing therapies based on protein composition. However, correctly capturing
the system behavior poses at least two major challenges: The approximations
made in the potential must be accurate enough representations of the real
system chemistry; and sufficient sampling of phase space is also required.

In order for MD simulations to be used as part of clinical decision support
systems, it is necessary that results can be obtained in a timely fashion.
Typically, interventions are made on a timescale of a few days or, at most, a
week. The necessity for rapid turn around times places additional demands on
simulation protocols which need to be optimized to gain results with a short
turn around time. Further to the scientific and practical considerations
outlined above, it is vital that reliable uncertainty estimates are
provided alongside all quantitative results for simulations to provide
actionable predictions.

We have developed a number of free energy calculation protocols based on the
use of ensemble molecular dynamics simulations with the aim of meeting these
requirements~\cite{Sadiq2008, Sadiq2010, Wan2017brd4, Wan2017trk}. Basing
these computations on the direct calculation of ensemble averages facilitates
the determination of statistically meaningful results along with complete
control of errors. The use of the ensemble approaches however, necessitates
the use of middleware to provide reliable coordination and distribution
mechanisms with low performance overheads.
