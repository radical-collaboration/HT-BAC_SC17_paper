
It is necessary to move beyond the prevailing paradigm of running individual
MD simulations, which provide irreproducible results and cannot provide
meaningful error bars~\cite{Bhati2017}. Further, the ability to flexibly
scale and adapt ensemble-based protocols to the systems of interest is vital
to produce reliable and accurate results on timescales which make it viable
to influence real world decision making. To meet these goals, we are
designing and developing the high-throughput binding affinity calculator
(HTBAC).

HTBAC employs the RADICAL-Cybertools 
% building blocks for workflow systems which can be used 
to build ensemble-based applications % like 
for executing protocols like ESMACS at scale and on leadership-class
machines.\jhanote{Careful not to imply that HTBAC is a workflow system.}
\jdnote{better?}\mtnote{Further iterated. Please review.} We show how the
ESMACS protocol \jhanote{check for consistency: HTBAC scales, or
implementation of ESMACS protocol scales?}\jdnote{done} scales almost
perfectly to hundreds of concurrent pipelines of binding affinity
calculations on Blue Waters. This permits a time-to-solution that is
essentially invariant of the size of candidate ligands as well as
the type and number of protocols concurrently employed. 

% \mtnote{I struggle to understand `This permits the rapid time-to-solution
% that is'} \jdnote{just reduced it to time-to-solution instead of rapid
% time- to-solution}

The use of software implementing well-defined abstractions like that of
``building blocks'', future proofs users of HTBAC to evolving hardware
platforms, while providing immediate benefits of scale and support for a
range of different application workflows. Thus, HTBAC represents an important
advance towards the use of molecular dynamics based free energy calculations
to the point where they can produce actionable results both in the clinic and
industrial drug discovery.

In the short term, the development of HTBAC will allow a significant increase
in the size of study. Much of the literature on MD-based free energy
calculations is limited to a few tens of systems, usually of similar drugs
bound to the same protein target. By facilitating investigations of much
larger datasets, HTBAC also provides a step towards tackling grand challenges
in drug design and precision medicine, where it is necessary to understand
the influences on binding strength for hundreds or thousands of drug-protein
variant combinations. Only in aiming to meet this ambitious goals we will be
able to reveal the limits of existing simulation technology and the
potentials used to approximate the chemistry of the real systems.

\footnotesize \textbf{Software and Data} HTBAC, Ensemble Toolkit and
RADICAL-Pilot can be found respectively at:
\url{https://github.com/radical-cybertools/htbac},
\url{https://github.com/radical-cybertools/radical.entk}, and
\url{https://github.com/radical-cybertools/radical.pilot}. Raw data and
scripts to reproduce experiments can be found at:
\url{https://github.com/radical-experiments/htbac-experiments.}

% \mtnote{Should we add a link to HTBAC code base?}
% \jdnote{Not sure, in case yes, }
