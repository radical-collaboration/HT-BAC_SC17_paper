HTBAC is initially designed to implement a single protocol using the EnTK
programming model to express the application logic of a binding affinity
protocol. Here, we exclusively focus on ESMACS to capture the workflow logic
and isolate the performance of a single protocol instance. HTBAC has been
extended as a Python library that enables the selection of multiple protocol
instances of ESMACS and TIES~\cite{dakka}.

The ESMACS protocol contains replicas 
\jhanote{I think replica is being used incorrectly. replica means copy. copy
of what? that remains undefined here and other places  replica is used.} which
perform a sequence of simulation stages for a given physical system. Each
pipeline in ESMACS consists of a sequence
of stages. Each stage consists of a task that performs unique functions,
including pre-processing and molecular dynamics simulations.
Fig~\ref{figure:HTBAC} shows how pipelines, stages and tasks are organized for
the ESMACS protocol. Each stage is composed of a single task with a set of
attributes that define parameters like the location of input files, the number
of simulations and the MD engine(s) used to launch those simulations. The
ESMACS protocol has 7 stages: the first, second and last stages perform
staging of the input/output data, the middle stages indicate simulation tasks.
A task is appended to a stage and stages are appended to a pipeline to
maintain temporal order during execution.

\begin{figure}
\centering
  \includegraphics[width=0.5\textwidth]{FIGURES/HTBAC_Workflow_ESMACS.pdf}
  \caption{ESMACS protocol implemented as an ensemble application, encoded
  using the EnTK API\@. A protocol represents a physical system and is
  encoded as a set of independent pipelines. Each pipeline maps to a single
  replica. ESMACS consists of 25 replicas. Stages within a pipeline are
  executed sequentially. Each stage contain a single task performing unique
  functions, as required by the protocol. Stages S3--S6 contain molecular
  dynamics simulation tasks executed with NAMD\@.}\label{figure:HTBAC}
\end{figure}

Fig.~\ref{figure:ht-bac_rp} shows how the ESMACS protocol integrates with
EnTK. \st{ESMACS consists of replicas where each replica maps to a pipeline.}
EnTK converts the set of pipelines into a set of tasks called compute unit
descriptions and submits them to RP\@. In addition, EnTK provides methods for
the user to specify a resource request including walltime, cores, queue, and
user credentials. EnTK convert this resource request into a pilot that RP
submits to a HPC machine. Once the pilot becomes active, it pulls compute unit
descriptions in bulk from a database, executing them on the pilot resources.

\begin{figure}
\centering
  \includegraphics[width=0.5\textwidth]{FIGURES/ht-bac-rp_integration.pdf}
  \caption{Integration between ESMACS and EnTK\@. Numbers indicate
  the temporal sequence of execution. The database (DB) of RADICAL-Pilot (RP)
  can be deployed on any host reachable from the resources. RP pushes compute
  units (CU) to DB and pulls them for execution.}\label{figure:ht-bac_rp}
\end{figure}

% Included in the HTBAC workflow expression, the user is able to define and
% assign a physical system to a specific protocol along with the number of
% replicas. ESMACS and TIES protocols differ in the details of the pipelines,
% stages and synchronization~\cite{Bhati2017}.

% \jhanote{Do you mean a workflow management system? Is it really a workflow
% (management) system? What are the essential properties of a workflow
% management system?}

% that sits between the user and cyberinfrastructure

% \jhanote{"sits between user and cyberinfrastructure" is too colloquial. is
% it a library? is it a runtime module?}

% in order to scale

% \jhanote{in order to scale: scale what?}

% and investigate free energy protocols with a variety of physical systems.
% HTBAC allows to encode binding free energy protocols, such as ESMACS and
% TIES, into ensemble applications\@. A protocol in HTBAC encoded as an
% ensemble of pipelines comprised of identical sequence of stages.

% \jhanote{ensemble of replicas is redundant. a replica is an ensemble
% member, and an ensemble is by definition comprised of ensemble members. the
% question is what is that ensemble member. here it is a pipeline of ... }

% \jhanote{the definition of protocol profferred here is too generic: e.g.,
% per definition of protocol above it is unclear if "interacting" replicas
% would be covered?}
% \jdnote{As per ESMACS and TIES, the replicas are non-interacting. Ensembles
% can be interacting but as per ESMACS/TIES protocols the replicas are not
% required to be interacting.} \jhanote{I think this is narrow/constraining:
% we have discussed how replica exchange might be a future protocol}

% We express the application logic of HTBAC using the user-facing API of EnTK
% (\S\ref{ssec:entk}). The EnTK API and its programming model allow HTBAC to
% express the workflows associated with different protocols as ensemble-based
% applications.

% \jhanote{lets get the definition and description clear, then we can come
% back and describe the advantages} \st{minimizing development effort and
% complexity.}

% directly to a set of pipelines in EnTK, where each pipeline contains
% functions that operate on a given replica. EnTK interprets these replicas
% as independent pipelines. Each pipeline consists of multiple stages
% representing a well-defined execution order; each stage can contain
% heterogeneous workloads. Although each stage of a pipeline depends on its
% predecessor, the pipelines execute independently of each other.

% ESMACS and TIES protocols differ in the details of the pipelines, stages
% and synchronization~\cite{Bhati2017}.

% \mtnote{I am afraid we need to iterate the whole pragraph. We need to
% separate between the abstracitons used in the ESMACS protocol (replica,
% function, simulation) to those of EnTK (pipeline, stage and task). Once
% separated, we need to map the former into the latter.} \jdnote{better? also
% see caption of HTBAC figure}

% The ESMACS protocol consist of pipelines with stages comprised of
% heterogeneous tasks. For example, equilibration and production, followed by
% post processing steps.

% \begin{figure}
% \centering
%   \includegraphics[width=0.5\textwidth]{FIGURES/HT-BAC_NAMD_pipelines_contr
%   ol_flow_only.pdf}
%   \caption{ESMACS protocol indicating how an N replica ensemble is
%   implemented in HTBAC. Each protocol instance is mapped to a single EnTK
%   pipeline. Each pipeline is equivalent and represents a set of simulations
%   which are captured as stages by EnTK.}\label{figure:ESMACS-pipelines}
% \end{figure}

%\begin{itemize}
% \item 1) Untar configuration files
% \item 2) Preprep
% \item 3) Minimize with decreasing restraints
% \item 4) Equilibration: NVT simulation at 50K, with restraints
% \item 5) Equilibration: NPT simulation at 300K, with decreasing restraints
% \item 6) Equlibratin: NPT at 300k, no constraints
% \item 7) Tarball output files
%\end{itemize}

% We capture the integration of the application (ESMACS protocol) and how it
% interfaces with EnTK in Fig.~\ref{figure:ht-bac_rp}. HTBAC provides methods
% for the user to specify a resource request including walltime, cores,
% queue, and user credentials. EnTK converts the HTBAC workflow into a set of
% tasks called compute unit descriptions and submits them to RP, along with
% the resource request. RP uses SAGA to submit a job to the specified queue
% in the batch system of the HPC machine. Once the job is scheduled by the
% batch system, the pilot becomes active, and it bootstaps the Agent module
% of RP\@. The Agent communicates with the MongoDB database (RP DB), and
% pulls compute unit descriptions in bulk. Once resources become available,
% the compute unit descriptions are translated into executable units and
% spawned for execution.
% \jhanote{I do not think the last few sentences in the paragraph above are
% relevant to HTBAC. A HTBAC user/developer used EnTK and should know nothing
% below that. As an HTBAC user I would be justified in not knowing anything
% about SAGA and RP!}
% \jdnote{We have Fig.~\ref{figure:ht-bac_rp} just below that describes the
% integration of HTBAC, EnTK, RP. We mention the Agent, RP DB, Tasks, CUs,
% Pipelines, etc. Therefore, I included a brief description of these
% components, or else the reader would be left wondering what these
% components are.}
% \jhanote{the reader will still be left wondering what an Agent is, what
% SAGA is. These are not described or discussed anywhere in the text!}
% \jdnote{I suggest we change the figure then, otherwise the reader will
% wonder what the Agent and RP DB are.}

% \jhanote{At the end of this section the reader still does not know what
% HTBAC is? Also, the reader knows what HTBAC enables, but not how.}
% \jdnote{Not entirely sure if I understand this comment, I added a line at
% the beginning of this section to see if this is the path you're referring
% to. Also added a sentence below to address how}\jhanote{My comment is that
% we've not still described whether HTBAC is a set of scripts or a library?
% client side software or server side? etc.}

% Included in the HTBAC workflow expression, the user is able to define and
% assign a physical system to a specific protocol along with the number of
% replicas.

% \jhanote{I think the workflow description includes the protocol, the number
% of replicas and physical system. These are not defined after the workflow.}
% \jdnote{yes, fixed}

% \jhanote{I would  move this to the opening paragraph of this section: }
% \jdnote{done}

% \jdnote{Better?}\mtnote{further iterated.}

% When the job gets scheduled by BW, the job is given resources, and the
% first thing done is to bootstrap the agent. The Agent is bootstrapped and
% the first component (data staging). It calls the RP DB and pulls in bulk
% the avaialble CUs checks the properties of input files. THe units are ready
% for scehdulidng, passes it to the queue of the scheduler, it pushes them to
% the queue of the agent scheduler. The executor manages one unit at a time.
% When there is a unit in the queue (inptu staging) and spawns it to the BW
% executor (APRUN). The executor (agent) waits for a the unit to return,
% passes the unit to the output staging, and informs the scheduler.

% \begin{figure}
% \centering
%   \includegraphics[width=0.5\textwidth]{FIGURES/HT-BAC_NAMD_pipelines_contr
%   ol_flow_only.pdf}
%   \caption{\bf NAMD Stages of HTBAC ESMACS protocol.}
%   \label{figure:ESMACS-pipelines}
% \end{figure}

% We define the client resource in Fig.~\ref{figure:ht-bac_rp} as the
% workload system---HTBAC\mtnote{I do not understand this sentence. Some
% problems I see with it: I do not see a definition if Fig. 4; what is a
% `client resource'? What is a `workload system' in this context?} which
% describes a set of replicas with ordered functions as pipelines with stages
% and tasks
% \mtnote{The rest of the sentence seems to use functions as previously done
% in the opening paragraphs?}. EnTK interprets these pipelines as a
% functional set of tasks \mtnote{I am not sure I understand what a
% `functional set of tasks' is} and generates the pilot description that
% contains the resource configuration of how to run the HTBAC workload. For
% the ESMACS protocol running on Blue Waters we define the runtime system,
% queue, and the pilot size. Once RADICAL-Pilot receives this new workload,
% it generates a pilot that submits placeholders to the queue \mtnote{A pilot
% is a resource placeholder. The sentence needs editing.}. Once the pilot %
% is activated becomes active, the RP-Agent submits the tasks in the form of
% compute units to the placeholders to begin execution \mtnote{RP-Agent is
% the pilot and therefore the resource placeholder. Let's discuss in person
% about RP execution model and how it is implemented by both RP modules and
% components}.

% RADICAL-Cybertools provides advanced resource management capabilities and,
% thereby delivers the necessary high-throughput capabilities
% required\mtnote{Required by?}. HTBAC is integrated with the EnTK component
% of RCT\@.\mtnote{I am not sure what we want to say with this short
% paragraph. Maybe we want to expand on it or comment it out?}

% \begin{figure}[ht]
% \centering
%  \includegraphics[width=0.5\textwidth]{FIGURES/entk_htbac_integration.pdf}
%   \caption{\bf Integration between HT-BAC workflow system and EnTK that
%   shows resource/application managers.}
%   \label{figure:ht-bac_entk}
% \end{figure}
