HTBAC by design represents binding free energy protocols, such 
ESMACS and TIES. We model a protocol using a series of 
simulation steps and assigning the number of replicas. 
We express the application logic of HTBAC using the user-facing API 
components of EnTK. EnTK provides a common API,
execution and programming model, thus allowing HTBAC to express the workflows
associated with different protocols uniformly, and thus minimize development
effort and complexity. We describe these components for the ESMACS
protocol\@. A protocol typically corresponds to a single physical
system. The concept of an ensemble in the ESMACS protocol maps directly to
a set of pipelines in EnTK, where each pipeline contains functions that operate
on a given replica. EnTK interprets these replicas as independent pipelines.
Each pipeline consists of multiple stages representing a well-defined execution
order; each stage can contain heterogeneous workloads. Although each stage of a
pipeline depends on its predecessor, the pipelines execute independently of each
other. The patterns within pipelines are identical and describe an ensemble of replica
simulations, as shown in Fig~\ref{figure:HTBAC}.


\begin{figure}
\centering
  \includegraphics[width=0.5\textwidth]{FIGURES/HTBAC_Workflow_ESMACS.pdf}
  \caption{ESMACS protocol implemented as an HTBAC workflow, encoded using the EnTK PST model. 
  Each protocol represents a physical system and is captured as a set of independent pipelines. 
  Each pipeline maps to a single replica. Stages within a pipeline maintain temporal 
  ordering. Each stage contains a single task. Stages S3-S6 contain NAMD tasks.}
  \label{figure:HTBAC}
\end{figure}

The ESMACS protocol consist of pipelines with stages comprised of heterogeneous tasks. 
For example, equilibration and production, followed by post processing steps. 
The different protocols differ in the details of the pipelines, stages and 
synchronization~\cite{Bhati2017}.


% \begin{figure}
% \centering
%   \includegraphics[width=0.5\textwidth]{FIGURES/HT-BAC_NAMD_pipelines_control_flow_only.pdf}
%   \caption{ESMACS protocol indicating how an N replica ensemble is implemented in HTBAC.
%   Each protocol instance is mapped to a single EnTK pipeline.
%   Each pipeline is equivalent and represents a set of simulations which are captured as stages by
%   EnTK.}\label{figure:ESMACS-pipelines}
% \end{figure}

%\begin{itemize}
%	\item 1) Untar configuration files
%	\item 2) Preprep
%	\item 3) Minimize with decreasing restraints
%	\item 4) Equilibration: NVT simulation at 50K, with restraints
%	\item 5) Equilibration: NPT simulation at 300K, with decreasing
%	restraints
%	\item 6) Equlibratin: NPT at 300k, no constraints
%	\item 7) Tarball output files
%\end{itemize}

Each stage is composed of a single unique task which is described by a set of
attributes that define the workload parameters such as the location of input
files, the number of simulations and the MD engine(s) to launch simulations.
The ESMACS protocol defines 7 stages, in which the first and last
stages perform staging of the input/output data. The middle stages indicate
simulation tasks as shown in Fig~\ref{figure:HTBAC}. The task is
appended to a stage and stages are appended to a pipeline to maintain
temporal order. The workflow relies on a resource configuration which
consists of the details required to use a resource where the application will
be executed including runtime, queue, and account details. We capture the
integration of the application (ESMACS protocol) and how it interfaces with
EnTK in Fig~\ref{figure:ht-bac_rp}.

% \begin{figure}
% \centering
%   \includegraphics[width=0.5\textwidth]{FIGURES/HT-BAC_NAMD_pipelines_control_flow_only.pdf}
%   \caption{\bf NAMD Stages of HTBAC ESMACS protocol.}
%   \label{figure:ESMACS-pipelines}
% \end{figure}


We define the client resource in Fig~\ref{figure:ht-bac_rp} as the workload
system---HTBAC which describes a series of replicas with ordered functions as
a pipelines with stages and tasks. EnTK interprets these pipelines as a
functional set of tasks and generates the pilot description that contains the
resource configuration of how to run the HTBAC workload. For the ESMACS
protocol running on Blue Waters we define the runtime system, queue, and the
pilot size. Once RADICAL-Pilot receives this new workload it generates a
pilot that submits placeholders to the queue. Once the pilot is activated,
the RP-Agent submits the tasks in the form of compute units to the
placeholders to begin execution.


\begin{figure}
\centering
  \includegraphics[width=0.5\textwidth]{FIGURES/ht-bac-rp_integration.pdf}
  \caption{Integration between HTBAC workflow and EnTK\@. Numbers
  indicate the temporal sequence of execution. The database (DB) of
  RADICAL-Pilot (RP) can be deployed on any host reachable from the
  resources. RP pushes compute units (CU) to DB and RP-Agent pulls them for
  execution. \dwwnote{I think CU needs to be defined
  here}\mtnote{Better?}}\label{figure:ht-bac_rp}
\end{figure}


RADICAL-Cybertools provides advanced resource management capabilities and,
thereby delivers the necessary high-throughput capabilities required. HTBAC is
integrated with the EnTK component of RCT. 


% \begin{figure}[ht]
% \centering
%  \includegraphics[width=0.5\textwidth]{FIGURES/entk_htbac_integration.pdf}
%   \caption{\bf Integration between HT-BAC workflow system and EnTK that shows resource/application managers.}
%   \label{figure:ht-bac_entk}
% \end{figure}
