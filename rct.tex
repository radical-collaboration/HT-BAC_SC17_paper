We have designed RADICAL-Cybertools (RCT) to address some of the challenges
in developing and executing workflows on HPC platforms, which are
functionally delineated middleware building blocks. The HTBAC workflow
\mtnote{I do not think HTBAC is a workflow! Should we just say `HTBAC uses
two RCT components\ldots'?} depends on two RCT components, mainly the
Ensemble Toolkit (EnTK) and RADICAL-Pilot (RP).

EnTK provides the ability to create and execute ensemble- based
workflows/applications with complex coordination and communication but
without the need for explicit resource management. EnTK uses RP as the
runtime system which provides resource management and task execution
capabilities.

RCT eschew the concept of a monolithic workflow systems and uses ``building
blocks''. RCT provide scalable implementations of building blocks in Python
that are used to support dozens of scientific applications on 
high-performance and distributed systems~\cite{turilli2016analysis, angius2017converging,treikalis2016repex, balasubramanian2016ensemble, balasubramanian2016extasy}. 
In this Section we discuss details of RP and EnTK, and understand how these
components have been used to support the flexible and scalable execution of
pipelines, which will permit a deeper understanding of HTBAC in the next
Section.

% ---------------------------------------------------------------------------
\subsection{RADICAL-Pilot}

The scalable execution of applications with large ensembles of tasks is
challenging. Traditionally, two methods are used to execute multiple tasks on
a resource: each task is scheduled as an individual job, or MPI capabilities
are used to execute multiple tasks as part of a single job. The former method
suffers from unpredictable queue time: each task independently awaits in the
resource's queue to be scheduled. The latter method relies on the fault
tolerance of MPI, and is suitable to execute tasks that are homogeneous and
have no interdependencies.

The Pilot abstraction~\cite{turilli2017comprehensive} solves these issues:
The pilot abstraction: (i) uses a placeholder job without any tasks assigned
to it, so as to acquire resources via the local resource management system
(LRMS); and, (ii) decouples the initial resource acquisition from
task-to-resource assignment. Once the pilot (container-job) is scheduled via
the LRMS, it can pull computational tasks for execution. This functionality
allows all the computational tasks to be executed directly on the resources,
without being queued via the LRMS\@. % individually. 
The pilot abstraction thus supports the requirements of task-level
parallelism and high-throughput as needed by science drivers, without
affecting or circumventing the queue policies of HPC resources.

RADICAL-Pilot is an implementation of the pilot abstraction, engineered to
support scalable and efficient launching of heterogeneous tasks across
different platforms.