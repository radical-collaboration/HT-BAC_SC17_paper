\subsection{RADICAL-Cybertools}

We discuss the RADICAL-Cybertools (RCT) software stack used support the
flexible and scalable execution of pipelines and upon which HTBAC is built.
RCT consists of three main components: Ensemble Toolkit~\cite {entk-icpp-2016}
provides the ability to create and execute ensemble-based
workflows/applications with complex coordination and communication but without
the need for explicit resource management. It uses another RCT component ---
RADICAL-Pilot~\cite{review_radicalpilot} which provides resource management
and task execution capabilities, which in turn uses RADICAL-SAGA~\cite{saga-x
,ogf-gfd-90} as an interoperability resource access layer.

RCT eschew the concept of a monolithic workflow systems in favor of a fresh
perspective to workflows using ``building blocks''. RCT provide scalable
implementations of building blocks in Python and are used to support dozens of
scientific applications on high-performance and distributed systems. RCT has
also been used extensively to support biomolecular sciences
algorithms/methods, e.g., replica-exchange and adaptive sampling.

\subsubsection{RADICAL-Pilot}\label{sec:pilot}

A primary challenge faced is the scalable execution of applications consisting
of large ensembles of tasks.  Traditionally, each task is submitted as an
individual job, or MPI capabilities are used to  execute multiple tasks as
part of a multi-node single job. The former method suffers from unpredictable
queue time for each job; the latter is suitable to execute tasks that are
homogeneous and have no dependencies, and relies on the fault tolerance of
MPI.

The Pilot abstraction~\cite{review_pilotreview} solves these issues:  The
pilot abstraction, (i) uses a container-job as a placeholder to acquire
resources via the local resource management system (LRMS) and,  (ii) to
decouple the initial resource acquisition from task-to-resource assignment.
Once the pilot (container-job) is scheduled via the LRMS, it can then directly
be populated with the computational tasks. This functionality allows all the
computational tasks to be executed directly on the resources, without being
queued at the LRMS individually. This approach thus supports the requirements
of task-level parallelism and high-throughput as needed by science drivers.

RADICAL-Pilot is an implementation of the pilot abstraction, engineered to
support scalable and efficient launching of heterogeneous tasks across
different platforms.