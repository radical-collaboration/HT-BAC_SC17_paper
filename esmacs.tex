In order to demonstrate the scalability and performance of HT-BAC we will focus on the ESMACS protocol alone applied to a representative kinase system.

\subsection{Binding Affinity Calculation Protocol}

We have demonstrated the lack of reproducibility of single trajectory
approaches in both HIV-1 protease and MHC systems, with calculations for the
same protein-ligand combination, with identical initial structure and force
field, shown to produce binding affinities varying by up to 12 kcal mol
$^{-1}$ for small ligands (flexible ligands can vary even more).
\cite{Wan2015, Sadiq2010, Wright2014}. Indeed, our work has revealed how
completely unreliable single simulation based approaches are. While accuracy
of force fields could be a source of error, we know from our work to date
\cite{} that the very large fluctuations in trajectory-based calculations
account for the lion’s share of the variance (hence also uncertainty) of the
results. Almost all MMPBSA studies in the literature use the so-called
1-trajectory method, in which the energies of protein-inhibitor complexes,
receptor proteins and ligands are extracted from the MD trajectories of the
complexes alone. ESMACS protocols can additionally use separate ligand and
receptor trajectories to account for adaptation energies. Previous work has
produced results in agreement with previously published experimental findings
\cite{Sadiq2010, Wan2011, Wright2014, Bhati2017, Wan2017brd4, Wan2017trk}, and
correctly predicted the results of experimental studies performed by
colleagues in collaboration \cite{Bunney2015}.


\subsubsection{ESMACS}

Two protocols of particular importance for XXXX are ESMACS (enhanced sampling
of molecular dynamics with approximation of continuum
solvent)\cite{Wan2017brd4} and TIES (thermodynamic integration with enhanced
sampling) \cite{Bhati2017}. The former is based on variants of the molecular
mechanics Poisson-Boltzmann surface area (MMPBSA) end-point method and the
latter the `alchemical' thermodynamic integration (TI) approach. In both cases
ensembles of MD simulations are employed in order to perform averaging and to
obtain tight control of error bounds in our estimates.

Each replica within the ESMACS protocol consists of a series of simulation
steps followed by post production analysis. Generally an ESMACS replica will
contain between 3 and 12 equilibration simulation steps followed by a
production MD run all of which are conducted in the NAMD package
\cite{Phillips2005}. The first step is system minimization, the following
steps involve the gradual release of positional constraints upon the structure
and the heating to a physiologically realistic temperature. Upon completion of
the MD simulation, free energy computation (via MMPBSA and potentially normal
mode analysis) is performed using AmberTools \cite{amber14, Case2005,
MillerIII2012}.

\subsection{Benchmark kinase system}

A common target of kinase inhibitors is the epidermal growth factor receptor (EGFR) which regulates important cellular processes including proliferation, differentiation and apoptosis.
EGFR is frequently over expressed in a range of cancers, and is associated with disease progression and treatment. 
Clinical studies have shown that EGFR mutations confer tumour sensitivity to tyrosine kinase inhibitors in patients with non-small-cell lung cancer.
The tyrosine kinase domain of EGFR contains 288 residues, the full simulation system including solvent and the AEE788 inhibitor contains approximately 50 thousand atoms.
The well established AMBER ff99SBildn and GAFF force fields \cite{Maier2015, Wang2004} were used to parameterize the system for this work.
