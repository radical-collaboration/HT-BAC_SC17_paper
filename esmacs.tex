In order to demonstrate the scalability and performance of HT-BAC we will focus on the ESMACS protocol alone applied to a representative kinase system.

\subsection{ESMACS}

Each replica within the ESMACS protocol consists of a series of simulation steps followed by post production analysis.
Generally an ESMACS replica will contain between 3 and 12 equilibration simulation steps followed by a production MD run all of which are conducted in the NAMD package \cite{Phillips2005}.
The first step is system minimization, the following steps involve the gradual release of positional constraints upon the structure and the heating to a physiologically realistic temperature.
Upon completion of the MD simulation, free energy computation (via MMPBSA and potentially normal mode analysis) is performed using AmberTools \cite{amber14, Case2005, MillerIII2012}.

\subsection{Benchmark kinase system}

A common target of kinase inhibitors is the epidermal growth factor receptor (EGFR) which regulates important cellular processes including proliferation, differentiation and apoptosis.
EGFR is frequently over expressed in a range of cancers, and is associated with disease progression and treatment. 
Clinical studies have shown that EGFR mutations confer tumour sensitivity to tyrosine kinase inhibitors in patients with non-small-cell lung cancer.
The tyrosine kinase domain of EGFR contains 288 residues, the full simulation system including solvent and the AEE788 inhibitor contains approximately 50 thousand atoms.
The well established AMBER ff99SBildn and GAFF force fields \cite{Maier2015, Wang2004} were used to parameterize the system for this work.
