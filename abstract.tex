Resistance to chemotherapy and molecularly targeted therapies is a major
factor in limiting the effectiveness of cancer therapies. In many cases
resistance can be linked to genetic changes in target proteins, either pre-
existing or evolutionarily selected during treatment. Key to overcoming this
challenge is the understanding of the molecular determinants of drug binding.
The key physical parameters determining drug effectiveness are the binding
free energy and the residence time of a ligand. Using molecular simulation we
can gain insights into both of these quantities, which can inform both
stratified or personal treatment regimes and drug development. The level of
sampling required to obtain accurate and reproducible binding affinities and
kinetic information can only efficiently be obtained in a timely fashion
through the use of ensembles of many simulations executed on HPC resources. In
state of the art calculation techniques each simulation may, in fact, consist
of a complex multi-stage workflow. The resulting trajectories reveal in atomic
detail how interactions and protein behaviour change as a result of mutations,
and account for the molecular basis of drug efficacy. We will study a wide
range of cancer drugs and candidate ligands in order to support personalised
clinical decision making based on genome sequencing and drug discovery, using
automated workflows underpinned by ensemble-based high performance computing
methods running at unprecedented scales. In order to do this, we have
developed the High-Throughput Binding Affinity Calculator (HT-BAC) which is
the integration of the BAC -- Binding Affinity Calculator -- developed by the
Centre for Computational Science at UCL, and RADICAL-Cybertools (in particular
EnTK). BAC represents the state-of-the-art in expressing biomedically
important workflows for rapid, accurate, precise and reliable free energy
based binding affinity calculations. RADICAL-Cybertools are a suite of
functional components that support the interoperable and scalable execution of
multiple simulations. HT-BAC provides the ability to separate the "setting-up"
(i.e., defining) of a workflow from its "execution".
