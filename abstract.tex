Resistance to chemotherapy and molecularly targeted therapies is a major
factor in limiting the effectiveness of cancer therapies. In many cases
resistance can be linked to genetic changes in target proteins, either pre-
existing or evolutionarily selected during treatment. Key to overcoming this
challenge is the understanding of the molecular determinants of drug binding.
The key physical parameters determining drug effectiveness are the binding
free energy and the residence time of a ligand. Using molecular simulation we
can gain insights into both of these quantities, which can inform both
stratified or personal treatment regimes and drug development. The level of
sampling required to obtain accurate and reproducible binding affinities and
kinetic information often require the use of an ensemble of a complex multi-
stage pipeline of simulations. The resulting trajectories reveal in atomic
detail how interactions and protein behavior change as a result of mutations,
and account for the molecular basis of drug efficacy. Further, there is a need
to study a wide range of cancer drugs and candidate ligands in order to
support personalized clinical decision making based on genome sequencing and
drug discovery. To support the scalable, adaptive and automated calculation of
the binding free energy on high-performance computing resources, we introduce
the High-throughput Binding Affinity Calculator (HTBAC). HTBAC uses a building
block approach to workflow systems in order to provide a trade-off between
flexibility and performance. We demonstrate close to perfect weak scaling to
hundreds of concurrent multi-stage pipelines. This permits the rapid time-to-
solution that is essentially invariant of the calculation protocol, size of
candidate ligands and number of ensemble simulations. As such, HTBAC advances
the state of the art of binding affinity calculation methods and protocols.

% using automated workflows underpinned by ensemble-based high performance
% computing methods running at unprecedented scales.

% HTBAC allows the investigation ....
% ... HTBAC provides the ability to separate the "setting-up" (i.e., defining) of a
% workflow from its "execution".


% which is the integration of the BAC -- Binding Affinity Calculator --
% developed by the Centre for Computational Science at UCL, and RADICAL-
% Cybertools (in particular EnTK). H

% HTBAC represents the state-of-the-art in expressing biomedically
% important workflows for rapid, accurate, precise and reliable free energy
% based binding affinity calculations. RADICAL-Cybertools are a suite of
% functional components that support the interoperable and scalable execution of
% multiple simulations. 

