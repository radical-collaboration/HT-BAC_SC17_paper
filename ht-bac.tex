\subsection{High throughput binding affinity calculator Workflow}

(Stefan/Jumana)

BAC uses the Ensemble Toolkit to create a high-throughput workflow which we refer to as HTBAC. The Ensemble Toolkit API exposes four components (‘Resource Handle’, ‘Pipeline’, ‘Stage’, and ‘Task') to the user which express the application logic of HTBAC. We describe these components for one of the HTBAC protocols, ESMACS. The ESMACS protocol contains a set of pipelines where each pipeline contains functions that operate on a given replica. The pattern of pipelines are identical and describes an ensemble of preprocessing and simulation stages as shown in figure control-flow. 


\begin{figure}[tb]
\centering
  \includegraphics[width=0.5\textwidth]{FIGURES/HT-BAC-NAMD-pipelines-control-flow-only.pdf}
  \caption{\bf ESMACS protocol indicating n-pipelines where each pipeline represents...}
   \label{figure:ESMACS-pipelines}
\end{figure}

\begin{itemize}
	\item 1) Untar configuration files
	\item 2) Preprep
	\item 3) Minimize with decreasing restraints
	\item 4) Equilibration: NVT simulation at 50K, with restraints
	\item 5) Equilibration: NPT simulation at 300K, with decreasing restraints 
	\item 6) Equlibratin: NPT at 300k, no constraints
	\item 7) Tarball output files 
\end{itemize}

Each stage is composed of a single unique task which is described by set of attributes that define the workload parameters such as the location of input files, the number of simulations and the MD engine to run the simulations. The task is appended to a stage and stages are appended to a pipeline to maintain temporal order. The workflow relies on a resource configuration which consists of the details required to use a resource where the application will be executed including runtime, queue, and account details. We capture the integration of the application (ESMACS protocol) and how it interfaces with EnTK in figures ht-bac-rp-integration and entk-htbac-integration. 

\begin{figure}[tb]
\centering
  \includegraphics[width=0.5\textwidth]{FIGURES/ht-bac-rp_integration.pdf}
  \caption{\bf Integration between HT-BAC workflow system and EnTK. Numbers indicate the temporal sequence of execution. RADICAL-Pilot (RP) database (DB) can be deployed on any host reachable from the resources.}
   \label{figure:ht-bac_rp}
\end{figure}

\begin{figure}[tb]
\centering
  \includegraphics[width=0.5\textwidth]{FIGURES/entk_htbac_integration.pdf}
  \caption{\bf Integration between HT-BAC workflow system and EnTK that shows resource/application managers.}
   \label{figure:ht-bac_entk}
\end{figure}

Add code snippets 

How to express ESMACS and how to execute ESMACS


Remind the reader of the 7 stages of the ESMACS protocol, and how the stages are expressed, and how the pipelines are added 


