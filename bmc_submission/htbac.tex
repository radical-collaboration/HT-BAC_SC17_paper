\subsection{ESMACS}\label{sec:htbac}

Initially, we designed HTBAC to implement a single binding affinity protocol,
using the EnTK programming model to express the application logic. Here, we
exclusively focus on ESMACS to capture the workflow logic and isolate the
performance of a single protocol instance. HTBAC has been extended as a
Python library that enables the selection of multiple protocol instances of
ESMACS and TIES~\cite{dakka}.

A simulation pipeline is a defined sequence of simulation stages for a given
physical system. In the ESMACS protocol, these simulation pipelines are
replicated, where replicas differ only by their parameter configurations,
namely initial velocities, which are randomly generated and assigned by NAMD
at the start of execution. A simulation pipeline in the ESMACS protocol has 7
stages: the first, second and last stages perform staging of the input/output
data, the middle stages indicate simulation tasks. A task is appended to a
stage and stages are appended to a pipeline to maintain temporal order during
execution.

% The ESMACS protocol performs a sequence of simulation stages for a given
% physical system. ESMACS replicates this sequence of simulation stages.
% Replicas differ only by their parameter configurations, namely initial
% velocities, which are randomnly generated and assigned by NAMD at the start
% of execution.

Each simulation pipeline replica maps to an independent EnTK pipeline. Each
pipeline consists of a sequence of stages, and each stage consists of a
single task that performs unique functions, including pre-processing and
molecular dynamics simulations. Fig~\ref{figure:HTBAC} shows how pipelines,
stages and tasks are organized for the ESMACS protocol. A task is composed of
a set of attributes that define parameters like the location of input files,
the number of simulations and the MD engine(s) used to launch those
simulations.

% \mtnote{There is some repetition in this paragraph about
% pipelines/stages/tasks and ESMACS\@.}\jdnote{better?}\mtnote{Waiting for
% the paragraph above to be iterated before editing this paragraph
% further.}\jhanote{please review.}\mtnote{Works for me.}

Fig.~\ref{figure:ht-bac_rp} shows how the ESMACS protocol integrates with
EnTK\@. EnTK converts the set of pipelines into a set of tasks called compute
unit descriptions and submits them to RP\@. In addition, EnTK provides
methods for the user to specify a resource request including walltime, cores,
queue, and user credentials. EnTK converts this resource request into a pilot
that RP submits to a HPC machine. Once the pilot becomes active, it pulls
compute unit descriptions in bulk from a database, executing them on the
pilot resources.