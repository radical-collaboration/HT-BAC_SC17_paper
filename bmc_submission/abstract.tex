\parttitle{Background}
Resistance to chemotherapy and molecularly targeted therapies is a major
factor in limiting the effectiveness of cancer treatment. In many cases,
resistance can be linked to genetic changes in target proteins, either
pre-existing or evolutionarily selected during treatment. Key to overcoming
this challenge is an understanding of the molecular determinants of drug
binding. Using multi-stage pipelines of molecular simulations we can gain
insights into the binding free energy and the residence time of a ligand,
which can inform both stratified and personal treatment regimes and drug
development.
% The level of sampling required to obtain accurate and reproducible binding
% affinities and kinetic information often require the use of an ensemble of
% a complex multi-stage pipeline of simulations. The resulting trajectories
% reveal in atomic detail how interactions and protein behavior change as a
% result of mutations, and account for the molecular basis of drug efficacy.
% Further, there is a need to study a wide range of cancer drugs and
% candidate ligands in order to support personalized clinical decision making
% based on genome sequencing and drug discovery.
To support the scalable, adaptive and automated calculation of the binding
free energy on high-performance computing resources, we introduce the
High-throughput Binding Affinity Calculator (HTBAC). HTBAC uses a building
block approach in order to attain both workflow flexibility and performance.

\parttitle{Results}
We demonstrate close to perfect weak scaling to hundreds of concurrent
multi-stage binding affinity calculation pipelines. This permits a rapid
time-to-solution that is essentially invariant of the calculation protocol,
size of candidate ligands and number of ensemble simulations. 
\parttitle{Conclusions}
As such, HTBAC advances the state of the art of binding affinity calculations 
and protocols.
HTBAC provides the platform to enable scientists to study a wide range of cancer drugs and
candidate ligands in order to support personalized clinical decision making
based on genome sequencing and drug discovery.
