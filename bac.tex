

\subsection{Automated binding affinity calculations}

Molecular dynamics is a well established computational methodology for
studying the time-evolution and conformational dynamics of a diverse array of
physicochemical systems at the molecular level, from which a whole host of
physical and chemical properties can be determined \cite{Karplus2005}. The
implementation of any physically realistic molecular simulation has, however,
always been an involved and multistage process, often requiring the scientist
to overcome a large manual overhead in the construction, preparation, and
execution protocols needed to complete a set of simulations as well as to
invoke various analysis protocols for determining desired properties 
post-production. Several tools have been been developed to automate theses steps
for the rapid, accurate and reproducible computation of binding free energies
of small molecules to their target proteins. For example, BAC\cite{Sadiq2008}
is a partially automated workflow system which comprises model building
(including incorporation of mutations into patient specific protein
models); run of ensembles of MD simulations, using a range of free energy
techniques; and statistical analysis.

% Two protocols of particular importance for XXXX are ESMACS (enhanced sampling
% of molecular dynamics with approximation of continuum
% solvent)\cite{Wan2017brd4} and TIES (thermodynamic integration with enhanced
% sampling) \cite{Bhati2017}. The former is based on variants of the molecular
% mechanics Poisson-Boltzmann surface area (MMPBSA) end-point method and the
% latter the `alchemical' thermodynamic integration (TI) approach. In both cases
% ensembles of MD simulations are employed in order to perform averaging and to
% obtain tight control of error bounds in our estimates. 

% We have demonstrated the lack of reproducibility of single trajectory
% approaches in both HIV-1 protease and MHC systems, with calculations for the
% same protein-ligand combination, with identical initial structure and force
% field, shown to produce binding affinities varying by up to 12 kcal mol
% $^{-1}$ for small ligands (flexible ligands can vary even more).
% \cite{Wan2015, Sadiq2010, Wright2014} Indeed, our work has revealed how
% completely unreliable single simulation based approaches are. While accuracy
% of force fields could be a source of error, we know from our work to date
% \cite{} that the very large fluctuations in trajectory-based calculations
% account for the lion’s share of the variance (hence also uncertainty) of the
% results. Almost all MMPBSA studies in the literature use the so-called
% 1-trajectory method, in which the energies of protein-inhibitor complexes,
% receptor proteins and ligands are extracted from the MD trajectories of the
% complexes alone. ESMACS protocols can additionally use separate ligand and
% receptor trajectories to account for adaptation energies. Previous work has
% produced results in agreement with previously published experimental findings
% \cite{Sadiq2010, Wan2011, Wright2014, Bhati2017, Wan2017brd4, Wan2017trk}, and
% correctly predicted the results of experimental studies performed by
% colleagues in collaboration \cite{Bunney2015}.

In this work we demonstrate an enhancement of BAC, called high throughput BAC
(HT-BAC), which builds upon the Ensemble Toolkit (and thus RADICAL-Pilot) to
create a flexible software framework for runtime execution and performance.
