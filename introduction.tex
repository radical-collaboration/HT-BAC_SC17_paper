Cancer is recognized as a major public health problem throughout the world and is the second most common cause of death in the United States. \cite{Siegel2016}
In recent years chemotherapy based on targetted kinase inhibitors (TKIs) has played an increasingly prominent role in the treatment of cancer.
The design of TKIs was inspired by modern genetic understanding of DNA, the cell cycle, and molecular signaling pathways and they have been developed to selectively inhibit kinases involved in signaling pathways 
controling growth and proliferation which often become dysregulated in cancers.
This targetting of specific cancers reduces the risk of damage to healthy cells and increases treatment success.
Currently 35 FDA-approved small molecule TKIs are in clinical use, and for the past decade, they have represented a significant fraction of the \$37 billion U.S. market for oncology drugs \cite{FDA, Zhao2014}.
Imatinib, the first of these of drugs, is partially credited for doubling survivorship rates in certain cancers \cite{Zhao2014, ACSreport}.
Unfortunately, the development of resistance to these drugs limits the amount of time that patients can derive benefits from their treatment. 
Resistance to therapeutics is responsible for more than 90\% of deaths in patients with metastatic cancer \cite{Longley2005}.

While drug resistance can emerge via multiple mechanisms, small changes to the chemical composition of the therapeutic target (known as mutations) drive drug resistance in many patients.
In some commonly targeted kinases such as Abl, these changes account for as many as 90\% of treatment failure \cite{Shah2002}.
There are two major strategies for countering this threat to treatment efficacy: tailoring the drug regimen received by a patient according to the mutations present in their particular cancer, and development of more advanced %second- or third-line
therapies that retain potency for known resistance mutations.
In both cases, future developments require insight into the molecular changes produced by mutations, as well as ways to predict their impact on drug binding on a timescale much shorter than is typically experimentally feasible.
This represents a grand challenge for computational approaches.

The rapidly decreasing cost of next-generation sequencing has led many cancer centers to begin deep sequencing patient tumors to identify the genetic alterations driving individual cancers, with the ultimate goal of making individualized therapeutic decisions based upon this data -- an approach termed \textit{precision cancer therapy}.
While several common (recurrent) mutations have been catalogued due to their ability to induce resistance or susceptibility to particular kinase inhibitors, the vast majority of clinically observed mutations are rare, essentially ensuring that it will be impossible that catalog-building alone will be sufficient for making therapeutic decisions about the majority of individual patient tumors.
Fortunately, concurrent improvements in computational power and algorithm design mean that reliably quantifying differences in binding strength from molecular simulation is now becoming a genuine possibility.
This provides the opportunity to use automated simulation workflows to supplement existing inductive (`Baconian') decision support systems with deductive (`Popperian') predictive modelling and drug ranking \cite{Marias2011, Sloot2009}.
Where existing systems based on statistical inference are inherently limited in their range of applicability by the existence of data from previous similar cases the addition of modelling allows evidence based decision making even in the absence of direct past experience.

The same molecular simulation technologies that can be employed to investigate
the origins of drug resistance can also be used to design new therapeutics.
Creating simulation protocols which meet the varying practical demands of
different applications (such as the rapid turn around times needed for
clinical decision support) whilst obtaining statistically meaningful results
represents a significant computational challenge. Furthermore, it is highly
likely that differences between the varied systems which might be investigated
will demand different strategies can be employed as studies progress. For
example in drug design programmes it is usual to need to rapidly screen
thousands of candidate compounds to filter out the worst binders before more
sensitive methods are required when refining the structure. 
Not all changes induced in proteins shape or behaviour are local to the drug binding 
site and in some cases simulation protocols will need to adjust to account for 
complex interactions between drugs and their targets within individual studies.


Recent work that used molecular simulations to provide input to machine learning 
models \cite{Ash2017} designed merely to distinguish the highly active from weak 
inhibitors of the ERK2 kinase required simulations of 87 compounds.
If we wish to build on such studies to help inform later stages of the drug 
discovery pipeline, in which much more subtle alterations are involved, it is 
likely a much larger number of simulations would be required.
This is before we begin to consider the influence of mutations or the selectivity 
of drugs to the more than five hundred different genes in the human kinome. 
\cite{Li2016} In order to tackle these grand challenges using molecular simulation is 
likely to require studies executing thousands of compute intensive runs.
In order to contemplate such a programme of work it is necessary to
have computational tools that facilitate the execution and coordination of
varied workloads without incurring serious performance overheads.



%\jhanote{Is there a need for differentiation across these thousands of
%candiate compounds, or must they by design be treated and subject to the same
%protocols?}

%\jhanote{SJ to introduce extreme scale/exascale systems and provide stronger motivation for HTBAC}

% The strength of drug binding is determined by a thermodynamic property known
% as the binding free energy (or binding affinity). One promising technology for
% estimating binding free energies and the influence of protein composition on
% them is molecular dynamics (MD). In order to study the influence of mutations
% on drug binding atomistically detailed models of the drug and target protein
% can be used as the starting point for MD simulations. The chemistry of the
% system is encoded in what is known as a potential. \cite{Karplus2002} In the
% parameterization of the models each atom is assigned a mass and a charge, with
% the chemical bonds between them modelled as springs with varying stiffness.
% Using Newtonian mechanics the dynamics of the protein and drug can be tracked
% and using the principles of statistical mechanics estimates of thermodynamic
% properties obtained from simulations of single particles. The potentials used
% in the simulations are completely under the control of the user which allows
% the user to manipulate the system in ways which would not be possible in
% experiments. A particularly powerful example of this is so called `alchemical'
% simulations in which the potential used in the simulation changes from that
% representing one system to another during execution — allowing the calculation
% of free energy differences between the two systems (such as those induced by a
% protein mutation).

% MD simulations can reveal how interactions change as a result of mutations,
% and account for the molecular basis of drug efficacy. This understanding can
% form the basis for structure based drug design as well as helping to target
% existing therapies based on protein composition. Major challenges however
% exist in correctly capturing the system behaviour. Not only is it necessary
% that the approximations made in the potential are accurate enough
% representations of the real system chemistry but sufficient sampling of phase
% space is also required.

% %something about sampling techniques%

% In order for MD simulations to be used as part of clinical decision support systems it is necessary that results can be obtained in a timely fashion.
% Typically interventions are made on a timescale of a few days or at most a week.
% This need for rapid turn around times places additional demands on simulation protocols which need to be optimised to gain results within a short turn around time.

% Further to the scientific and practical considerations outlined above, in order for simulations to provide actionable predictions it is vital that robust and reliable uncertainty estimates are provided alongside all quantitative results.
% We have developed a number of free energy calculation protocols based on the use of ensemble molecular dynamics simulations with the aim of meeting these requirements \cite{Sadiq2008, Sadiq2010, Wan2017brd4, Wan2017trk}.
% Basing these computations on the direct calculation of ensemble averages facilitates the determination of statistically meaningful results along with complete control of errors.
% The consequence of the use of the ensemble approach is the need to automate the simulation, analysis and data transfer of a large number of simulations.
% This computational pattern necessitates the use of middleware that provides reliable coordination and distribution mechanisms with low performance overheads.

