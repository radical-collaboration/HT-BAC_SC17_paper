In recent years, chemotherapy based on targeted kinase inhibitors (TKIs) has
played an increasingly prominent role in the treatment of cancer. TKIs have
been developed to selectively inhibit kinases involved in the signaling
pathways that control growth and proliferation, which often become
dysregulated in cancers. This targeting of specific cancers reduces the risk
of damage to healthy cells and increases treatment success. Currently, 35
FDA-approved small molecule TKIs are in clinical use, and they represent a
significant fraction of the \$37 billion U.S. market for oncology
drugs~\cite{FDA, Zhao2014}. Imatinib, the first of these of drugs, is
partially credited for doubling survivorship rates in certain
cancers~\cite{Zhao2014, ACSreport}.

Unfortunately, the development of resistance to these drugs limits the amount
of time that patients can derive benefits from their treatment. Resistance to
therapeutics is responsible for more than 90\% of deaths in patients with
metastatic cancer~\cite{Longley2005}. While drug resistance can emerge via
multiple mechanisms, small changes to the chemical composition of the
therapeutic target (known as mutations) control treatment sensitivity and
drive drug resistance in many patients (see Fig.~\ref{fig:egfr}). In some
commonly targeted kinases such as Abl, these changes account for as many as
90\% of treatment failure~\cite{Shah2002}.

There are two major strategies for countering the threat to treatment
efficacy posed by resistance: tailoring the drug regimen received by a
patient according to the mutations present in their particular cancer, and
developing more advanced therapies that retain potency for known resistance
mutations. In both cases, future developments require insight into the
molecular changes produced by mutations, as well as ways to predict their
impact on drug binding on a timescale much shorter than is typically
experimentally feasible. This represents a grand challenge for computational
approaches.

The rapidly decreasing cost of next-generation sequencing has led many cancer
centers to begin deep sequencing of patient tumors to identify the genetic
alterations driving individual cancers. The ultimate goal is to make
individualized therapeutic decisions based upon these data---an approach
termed \textit{precision cancer therapy}. While several common (recurrent)
mutations have been cataloged for their ability to induce resistance or for
their susceptibility to particular kinase inhibitors, the vast majority of
clinically observed mutations are rare. Essentially, this ensures that it
will be impossible to make therapeutic decisions about the majority of
individual patient tumors by using catalog-building alone.

Fortunately, concurrent improvements in computational power and algorithm
design are enabling the use of molecular simulations to reliably quantify
differences in binding strength. This provides the opportunity to use
advances in molecular simulations to supplement existing inductive decision
support systems with deductive predictive modeling and drug
ranking~\cite{Marias2011, Sloot2009}. Where existing systems based on
statistical inference are inherently limited in their range of applicability
by the existence of data from previous similar cases, the addition of
modeling allows evidence based decision making even in the absence of direct
past experience.

The same molecular simulation technologies that can be employed to
investigate the origins of drug resistance can also be used to design new
therapeutics. Creating simulation protocols which have well defined
uncertainty and produce statistically meaningful results represents a
significant computational challenge. Furthermore, it is highly likely that
differences among investigated systems will demand different protocols as
studies progress. For example, drug design programmes often require the rapid
screening of thousands of candidate compounds to filter out the worst binders
before using more sensitive methods to refine the structure. Not all changes
induced in protein shape or behavior are local to the drug binding site and,
in some cases, simulation protocols will need to adjust to account for
complex interactions between drugs and their targets within individual
studies.

Recent work that used molecular simulations to provide input to machine
learning models~\cite{Ash2017} required simulations of 87 compounds even if
they were designed merely to distinguish the highly active from weak
inhibitors of the ERK2 kinase. If we wish to build on such studies to help
inform later stages of the drug discovery pipeline, in which much more subtle
alterations are involved, it is likely a much larger number of simulations
will be required. This is before we begin to consider the influence of
mutations or the selectivity of drugs to the more than 500 different
genes in the human kinome~\cite{Li2016}.

For molecular simulations to make the necessary impact, the dual challenge of
scale (thousands of concurrent multi-stage pipelines) and sophistication
(adaptive selection of binding affinity protocols based upon statistical
errors and uncertainty) will need to be tackled. Tools that facilitate the
scalable and automated computation of varied binding free energy calculations
on high-performance computing resources are necessary. To achieve that goal,
we introduce the High-throughput Binding Affinity Calculator~(HTBAC). HTBAC
translates recent advances in workflow system building blocks to the
application of rapid and accurate calculation of  binding affinities. We
demonstrate how the scaling of hundreds of concurrent binding affinity
calculations on a leadership class machine.
\mtnote{Do we want to change this?}\jhanote{It is wrong at a couple of levels:
in addition to not investigating/demonstrating the scaling of HTBAC, we have
not defined pipelines so far, thus the reader will not know what is scaling?
Proposed alternative above.} This permits the rapid time-to-solution that is
essentially invariant of the size of candidate ligands as well as the type and
number of protocols concurrently employed.

In the next Section, we provide details of ensemble molecular dynamics
approach and its advantages over the single trajectory approach. We also
introduce the ESMACS and related protocols to compute binding affinities using
ensemble based approaches. In Section~\ref{sec:3}, we discuss the
computational challenges associated with the scalable execution of multiple,
and possibly concurrently executing protocols. Section~\ref{sec:4} introduces
RADICAL-Cybertools---a suite of building blocks to address the challenges
outlined in Section~\ref{sec:3}---and describe how they are used by HTBAC to
manage the execution of binding affinity calculations at extreme scales.
Experiments to characterize the performance overheads of RADICAL-Cybertools
and the weak scaling properties of the HTBAC implementation of the ESMACS
protocol on the Blue Waters supercomputer are discussed in
Section~\ref{sec:6}. \mtnote{Do we want to change this? Alaternative:
``Experiments to characterize the performance and scalability of RADICAL-
Cybertools and HTBAC''.}\jhanote{Please see modifications above.} We conclude
with a discussion of the impact of HTBAC, implication for binding affinity
calculations and near-term future work.