Targetted kinase inhibitors play an increasingly prominent role in the treatment of cancer.
These therapeutics have been developed to selectively inhibit kinases involved in signaling pathways that control growth and proliferation, which often become dysregulated in cancers.
Currently 35 FDA-approved small molecule targeted kinase inhibitors are in clinical use, and for the past decade, they have represented a significant fraction of the \$37 billion U.S. market for oncology drugs \cite{FDA, Zhao2014}.
Imatinib, the first of these of drugs, is partially credited for doubling survivorship rates in certain cancers \cite{Zhao2014, ACSreport}.
Unfortunately, the development of resistance to these drugs limits the amount of time that patients can derive benefits from their treatment. 
Resistance to therapeutics is responsible for more than 90\% of deaths in patients with metastatic cancer \cite{Longley2005}. 
While drug resistance can emerge via multiple mechanisms, mutations in the therapeutic target drive drug resistance in many patients; in some commonly targeted kinases such as Abl, missense mutations are the mechanism of resistance in as many as 90\% of cases \cite{Shah2002}.
There are two major strategies for countering this threat to treatment efficacy: tailoring the drug regimen received by a patient according to the mutations present in their particular cancer, and development of more advanced second- or third-line therapies that retain potency for known resistance mutations.
In both cases, future developments require insight into the molecular changes produced by mutations, as well as ways to predict their impact on drug binding on a timescale much shorter than is typically experimentally feasible.
This represents a grand challenge for computational approaches.

The rapidly decreasing cost of next-generation sequencing has led many cancer centers to begin deep sequencing patient tumors to identify the genetic alterations driving individual cancers, with the ultimate goal of making individualized therapeutic decisions based upon this data—an approach termed \textit{precision cancer therapy}.
While several common (recurrent) mutations have been catalogued due to their ability to induce resistance or susceptibility to particular kinase inhibitors, the vast majority of clinically observed mutations are rare, essentially ensuring that it will be impossible that catalog-building alone will be sufficient for making therapeutic decisions about the majority of individual patient tumors.
Fortunately, concurrent improvements in computational power and algorithm design mean that reliably quantifying binding free energy differences from molecular simulation is now becoming a genuine possibility.
This provides the opportunity to use automated simulation workflows to supplement existing inductive (‘Baconian’) decision support systems with deductive (‘Popperian’) predictive modelling and drug ranking \cite{Marias2011, Sloot2009}.

In order for simulations to provide actionable predictions, it is vital that simulations not only accurately capture the chemistry of a system but that robust and reliable uncertainty estimates are provided.
We have developed a number of free energy calculation protocols based on the use of ensemble molecular dynamics simulations with the aim of meeting these requirements \cite{Sadiq2008, Sadiq2010, Wan2017brd4, Wan2017trk}.
Basing these computations on the direct calculation of ensemble averages facilitates the determination of statistically meaningful results along with complete control of errors.
The consequence of the use of the ensemble approach is the need to automate the simulation, analysis and data transfer of a large number of simulations.
This computational pattern necessitates the use of middleware that provides reliable coordination and distribution mechanisms with low performance overheads.

